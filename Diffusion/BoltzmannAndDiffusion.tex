\documentclass[12pt]{article}
%%%%%%%%%%%%%%%%%%%%%%%%%%%%%%%%%%%%%%%%%%%%%%%%%%%%%%%%%%%%%
%% Document setup for Syllabus and HW

\textwidth=7in
\textheight=9.5in
\topmargin=-1in
\headheight=0in
\headsep=.5in
\hoffset  -.85in

\usepackage{amsmath} % for \text{}
\usepackage{amsfonts} % For \text{}
\usepackage{amssymb} % For \text{}
\usepackage{amsthm}
\usepackage{bigints}

%% Stolen from Nathan Baker
\newcommand{\erf}{{\mathrm{erf}}}
\newcommand{\erfc}{{\mathrm{erfc}}}
\newcommand{\erfi}{{\mathrm{erfi}}}
\newcommand{\argh}{{\mathrm{arg}}}
\newcommand{\atan}{{\mathrm{atan}}}
\newcommand{\acos}{{\mathrm{acos}}}
\newcommand{\asin}{{\mathrm{asin}}}
\newcommand{\mat}[1]{\,\underline{\underline{#1}}\,}
\newcommand{\abs}[1]{\left| #1 \right|}
\newcommand{\norm}[1]{\left\| #1 \right\|}
\newcommand{\order}[1]{{\mathcal{O}} \left( #1 \right)}
\newcommand{\op}[1]{{\mathcal{#1}}}
\newcommand{\myfile}[1]{\texttt{#1}}
\newcommand{\myvar}[1]{\textsf{#1}}
\newcommand{\mean}[1]{{\left\langle {#1} \right\rangle}}

% The master boolean -- set this to "masterfalse" or "mastertrue"
%\newif\ifmaster \masterfalse
\newif\ifmaster \mastertrue

\newcommand{\hidesolution}[1]{ \ifmaster { {#1} } \else { {Go to class.} } \fi }
\newcommand{\hideanswer}[1]{ \ifmaster { #1 } \else {} \fi }
\newcommand{\solution}[1]{ \begin{proof}[Solution] \hidesolution{#1} \end{proof} }
\newcommand{\answer}[1]{ \hideanswer{\begin{proof}[Answer] #1 \end{proof}} }

%% MGL
\newcommand{\CC}{\mathbb{C}}
\newcommand{\N}{\mathbb{N}}
\newcommand{\Ham}{\mathcal{H}}
\newcommand{\sL}{\mathcal{L}}
\newcommand{\sP}{\mathcal{P}}
\newcommand{\sV}{\mathcal{V}}

%\newcommand{\vect}[1]{\,\vec{\mathbf{#1}}\,}
\newcommand{\vect}[1]{\,\mathbf{#1}\,}
\usepackage{tabularx}
\newcommand{\vtau}{\vect{\tau}}
\newcommand{\vomega}{\vect{\omega}}
\newcommand{\va}{\vect{a}}
\newcommand{\vb}{\vect{b}}
\newcommand{\vc}{\vect{c}}
\newcommand{\vd}{\vect{d}}
\newcommand{\ve}{\vect{e}}
\newcommand{\vf}{\vect{f}}
\newcommand{\vg}{\vect{g}}
\newcommand{\vh}{\vect{h}}
\newcommand{\vi}{\vect{i}}
\newcommand{\vj}{\vect{j}}
\newcommand{\vk}{\vect{k}}
\newcommand{\vl}{\vect{l}}
\newcommand{\vm}{\vect{m}}
\newcommand{\vn}{\vect{n}}
\newcommand{\vo}{\vect{o}}
\newcommand{\vp}{\vect{p}}
\newcommand{\vq}{\vect{q}}
\newcommand{\vr}{\vect{r}}
\newcommand{\vs}{\vect{s}}
\newcommand{\vt}{\vect{t}}
\newcommand{\vu}{\vect{u}}
\newcommand{\vv}{\vect{v}}
\newcommand{\vw}{\vect{w}}
\newcommand{\vx}{\vect{x}}
\newcommand{\vy}{\vect{y}}
\newcommand{\vz}{\vect{z}}

\newcommand{\vA}{\vect{A}}
\newcommand{\vB}{\vect{B}}
\newcommand{\vC}{\vect{C}}
\newcommand{\vD}{\vect{D}}
\newcommand{\vE}{\vect{E}}
\newcommand{\vF}{\vect{F}}
\newcommand{\vG}{\vect{G}}
\newcommand{\vH}{\vect{H}}
\newcommand{\vI}{\vect{I}}
\newcommand{\vJ}{\vect{J}}
\newcommand{\vK}{\vect{K}}
\newcommand{\vL}{\vect{L}}
\newcommand{\vM}{\vect{M}}
\newcommand{\vN}{\vect{N}}
\newcommand{\vO}{\vect{O}}
\newcommand{\vP}{\vect{P}}
\newcommand{\vQ}{\vect{Q}}
\newcommand{\vR}{\vect{R}}
\newcommand{\vS}{\vect{S}}
\newcommand{\vT}{\vect{T}}
\newcommand{\vU}{\vect{U}}
\newcommand{\vV}{\vect{V}}
\newcommand{\vW}{\vect{W}}
\newcommand{\vX}{\vect{X}}
\newcommand{\vY}{\vect{Y}}
\newcommand{\vZ}{\vect{Z}}
%\newcommand{\pd}[2]{\frac{\partial#1}{\partial#2}}
%\newcommand{\dd}[2]{\frac{d#1}{d#2}}
\newcommand{\dinline}[2]{d #1/d #2} % for derivatives
\newcommand{\pdinline}[2]{\partial#1/\partial#2}

\newcommand{\boldcent}[1] {\begin{center}\textbf{ #1 }\end{center}}
% ***********************************************************
% ******************* PHYSICS HEADER ************************
% ***********************************************************
% From http://www.dfcd.net/articles/latex/latex.html
% Version 2
%\documentclass[11pt]{article} 
\usepackage{amsmath} % AMS Math Package
\usepackage{amsthm} % Theorem Formatting
\usepackage{amssymb}	% Math symbols such as \mathbb
\usepackage{graphicx} % Allows for eps images
\usepackage{multicol} % Allows for multiple columns
%\usepackage[dvips,letterpaper,margin=0.75in,bottom=0.5in]{geometry}
% % Sets margins and page size
%\pagestyle{empty} % Removes page numbers
%\makeatletter % Need for anything that contains an @ command 
%\renewcommand{\maketitle} % Redefine maketitle to conserve space
%{ \begingroup \vskip 10pt \begin{center} \large {\bf \@title}
%	\vskip 10pt \large \@author \hskip 20pt \@date \end{center}
%  \vskip 10pt \endgroup \setcounter{footnote}{0} }
%\makeatother % End of region containing @ commands
\renewcommand{\labelenumi}{(\alph{enumi})} % Use letters for enumerate
% \DeclareMathOperator{\Sample}{Sample}
\let\vaccent=\v % rename builtin command \v{} to \vaccent{}
\renewcommand{\v}[1]{\ensuremath{\mathbf{#1}}} % for vectors
\newcommand{\gv}[1]{\ensuremath{\mbox{\boldmath$ #1 $}}} 
% for vectors of Greek letters
\newcommand{\uv}[1]{\ensuremath{\mathbf{\hat{#1}}}} % for unit vector
\newcommand{\hx}{\uv{x}}
\newcommand{\hy}{\uv{y}}
\newcommand{\hz}{\uv{z}}
\newcommand{\hr}{\uv{r}}
\newcommand{\hphi}{\uv{\phi}}
\newcommand{\htheta}{\uv{\theta}}
%\newcommand{\abs}[1]{\left| #1 \right|} % for absolute value
% overbar taken from http://tex.stackexchange.com/questions/22100/the-bar-and-overline-commands
\newcommand{\overbar}[1]{\mkern 1.5mu\overline{\mkern-1.5mu#1\mkern-1.5mu}\mkern 1.5mu}
\newcommand{\avg}[1]{\overbar{#1}} % for average
\newcommand{\avgb}[1]{\left< #1\right>} % for average
%\newcommand{\avg}[1]{\left< #1 \right>} % for average

%\let\underdot=\d % rename builtin command \d{} to \underdot{}
%\renewcommand{\d}[2]{\frac{d #1}{d #2}} % for derivatives
\newcommand{\fd}[2]{\frac{d #1}{d #2}} % for derivatives
\newcommand{\dd}[2]{\frac{d^2 #1}{d #2^2}} % for double derivatives
\newcommand{\pd}[2]{\frac{\partial #1}{\partial #2}} 
% for partial derivatives
\newcommand{\pdd}[2]{\frac{\partial^2 #1}{\partial #2^2}} 
\newcommand{\pddx}[3]{\frac{\partial^2 #1}{\partial #2 \partial #3}} 
% for double partial derivatives
\newcommand{\pdc}[3]{\left( \frac{\partial #1}{\partial #2}
 \right)_{#3}} % for thermodynamic partial derivatives
\newcommand{\ket}[1]{\left| #1 \right>} % for Dirac bras
\newcommand{\bra}[1]{\left< #1 \right|} % for Dirac kets
\newcommand{\braket}[2]{\left< #1 \vphantom{#2} \right|
 \left. #2 \vphantom{#1} \right>} % for Dirac brackets
\newcommand{\matrixel}[3]{\left< #1 \vphantom{#2#3} \right|
 #2 \left| #3 \vphantom{#1#2} \right>} % for Dirac matrix elements
\newcommand{\grad}[1]{\gv{\nabla} #1} % for gradient
\let\divsymb=\div % rename builtin command \div to \divsymb
\renewcommand{\div}[1]{\gv{\nabla} \cdot #1} % for divergence
\newcommand{\curl}[1]{\gv{\nabla} \times #1} % for curl
\newcommand{\lap}[1]{\gv{\nabla}^2 #1} % for laplacian
\let\baraccent=\= % rename builtin command \= to \baraccent
\renewcommand{\=}[1]{\stackrel{#1}{=}} % for putting numbers above =
\newtheorem{prop}{Proposition}
\newtheorem{thm}{Theorem}[section]
\newtheorem{lem}[thm]{Lemma}
\theoremstyle{definition}
\newtheorem{dfn}{Definition}
\theoremstyle{remark}
\newtheorem*{rmk}{Remark}

% ***********************************************************
% ********************** END HEADER *************************
% ***********************************************************

%%%%%%%%%%%%%%%%%%%%%%%%%%%%%%%%%%%%%%%%%%%%%%%%%%%%%%%%%%%%%

%\pagestyle{empty}
\usepackage{tikz}
\usepackage{filecontents,pgfplots}
\usepackage{enumerate}
\usepackage{nth}
\usepackage{wrapfig}
\usetikzlibrary{scopes}
%\usepackage{natbib}

\begin{document}
\setlength{\unitlength}{1in}
\title{Diffusion and Boltzmann's constant}
\author{Michael G. Lerner}
\date{\today}
\maketitle
%\begin{abstract}
%\end{abstract}

We'll use modern microscopy techniques to measure Boltzmann's constant
by looking at the diffusion of microspheres suspended in fluid and by
looking at the diffusion of quantum dots attached to PE lipids in
supported membranes. We will follow the methodology from
\cite{nakroshis2003measuring}. The paper is pretty good, and we'll do
both types of analysis mentioned in the article, meaning that we'll
want to use both equation (3) and equation (5) from the paper. Here's
where those come from. This treatment is known as Stokes-Einstein
diffusion.
\section{The Langevin Equation}

We've spent a lot of the class so far learning the lesson of
statistical physics: we can determine an enormous amount of physics
related to ``large'' systems by looking at average properties. Here,
we apply the same thinking to dynamics. If we want to describe the
motion of a grain of pollen floating on a pond (or a microsphere in
solution, or a lipid in a membrane), we know that we'll need some sort
of differential equation. The statistical physics trick is that we'll
want a \textit{stochastic} differential equation. That is, we'll want
our differential equation to have a \textit{random} term representing
the interaction of our pollen with the water. The specific type of
differential equation we'll want is called a Langevin equation. Don
Lemons has written a fantastic introduction to stochastic processes
\cite{LemonsStochasticIntro} and MacDonald \cite{MacDonaldNoise} has a
wonderful small book as well if this topic catches your interest. Reif
\cite{ReifStatMech} also has a nice derivation starting in \S15.5 and
we follow it below.

We'll treat a one-dimensional system, and the first goal is to get the
Langevin equation

\begin{equation}
  \label{Langevin}
  m\fd{v}{t} = -\zeta v + \eta(t)
\end{equation}

In looking at (\ref{Langevin}), the two terms on the right hand side
are new to us. The first of these a) slows us down and b) is
proportional to velocity, and so is known as a ``velocity-dependent
drag force'' and $\zeta$ is called the ``drag coefficient''. Although
we won't use it here, the quantity $1/\zeta$ is called the
``mobility'' and is typically written as $\mu$.
The second is our stochastic, or ``noise'' term, a rapidly fluctuating
force due to the environmental water molecules (the Greek letter eta
is chosen because $\eta$ looks like the ``n'' from ``noise'').

In general, we start with Newton's second law:

\begin{equation}
  \label{N2L} 
  m\dd{x}{t} = m\fd{v}{t} = F
\end{equation}
and we will try to say some smart things about $F$ that will let us
write (\ref{N2L}) as (\ref{Langevin}).

On the statistical side, if $F$ is the force on our microsphere due to
the surrounding waters, we'd like to consider the average of a large
number $N$ of identically prepared systems, finding the average force
at a specific time $t_1$ on our particle as

\begin{equation}
  \avgb{F(t_1)} \equiv \frac{1}{N}\sum_{n=1}^NF^{(n)}(t_1)
\end{equation}
where $n$ indexes the identically prepared systems. If we assume that
$F(t)$ is rapidly fluctuating, it must then be true that $v$ is also
rapidly fluctuating. 

The physical insight here is that we can then break $v$ up into two pieces:

\begin{equation}
  \label{vphysicalinsight}
  v = \avgb{v} + \tilde{v}
\end{equation}
where $\avgb{v}$ is the (slowly varying) ensemble average velocity, and
$\tilde{v}$ a rapidly fluctuating term with mean zero. The key thing
here is that, if we want the long-time behavior of our particle, the
relative \textit{magnitudes} of $\avgb{v}$ and $\tilde{v}$ don't
matter. In fact, the microscopic world is turbulent and extremely
violent. Living in the macroscopic world, we don't usually think about
such things, but the forces bombarding very small objects result in
$\tilde{v}$'s that are typically much larger than $\avgb{v}$'s.

%%%In any case, to get our description of $v$ (\ref{Langevin}), let's
%%%integrate over some time interval $\tau$:
%%%
%%%\begin{equation}
%%%  \label{tauint}
%%%  m\left[ v(t+\tau) - v(t)\right] = \int_t^{t+\tau}F(t')dt'
%%%\end{equation}
%%%
%%%where $t'$ is some dummy variable that disappears after the
%%%integration. We have to be a little careful in choosing $\tau$. A
%%%fuller description may be found in the references, but our physical insight was
%%%that there was a short (microscopic) time scale and a long
%%%time scale (macroscopic). $\tau$ must then be small on the macroscopic
%%%timescale for our description to work, but big enough on the
%%%microscopic time scale that we can use our physical insight equation
%%%(\ref{vphysicalinsight}). In practice, this is easy enough to do because the
%%%timescales are very far apart.
%%%
We can use the same sort of physical insight to the force acting on
the particle:

\begin{equation}
  \label{fphysicalinsight}
  F = \avgb{F} + \tilde{F} \equiv \avgb{F} + \eta
\end{equation}

We're almost done at this point, so take a quick look back at
(\ref{Langevin}) and (\ref{N2L}). We have the rapidly-varying part,
but we need to be a little more detailed in our treatment of the
slowly-varying part. We don't usually talk about force as
a function of velocity, but there's no reason not to. In fact,
%%%(\ref{tauint}) and (\ref{N2L}) make it quite clear that it's
(\ref{N2L}) makes it quite clear that it's
reasonable to do so.

If $\avgb{F}$ is the slowly-varying part of $F$, it must be a function
of the slowly-varying part of $v$, $\avgb{v}$. We know this because,
in equilibrium, if $\avgb{v}$ is zero, must find $\avgb{F}$ = 0. Being
Physicists, we immediately think to expand $\avgb{F(\avgb{v})}$ (for
the next equation, we'll use overbars to represent averages, because
$\avg{F}(\avg{v})$ looks much clearer than $\avgb{F(\avgb{v})}$) in a power series:

\begin{equation}
  \label{fmac}
  \avg{F}(\avg{v}) = \sum_{k=0}^\infty a_0 + a_1\avg{v} + a_2\avg{v}^2 + a_3\avg{v}^3 + ...
\end{equation}

If we assume that our system is unbiased (that is, that the constant
term in (\ref{fmac}) is zero), our first non-zero term is the linear
term. Renaming $a_1$ to be $-\zeta$ where the explicit minus sign
indicates that our environmental forces will tend to slow an object
down, we have our velocity-dependent drag force\footnote{Actually, the
derivation here is sweeping some things under the rug. Langevin's 1908
derivation just \textit{assumed} that one could model the average
force from the environment as a velocity-dependent drag. That's a
perfectly physical reasonable assumption on its own. More rigorously,
one can derive the generalized Langevin equation (GLE, see, for
example, \cite{TuckermanStatMechSim}), and from it
the Langevin equation used here.}:

\begin{equation}
  \label{vddrag}
  \avgb{F} = -\zeta \avgb{v}
\end{equation}
Plugging \ref{vddrag} into the right hand side of
\ref{fphysicalinsight}, and plugging the result into the right hand
side of \ref{N2L}, we get

\begin{equation*}
  m\dd{x}{t} = m\fd{v}{t} = F = \avgb{F} + \eta = -\zeta \avgb{v} + \eta
\end{equation*}
which, looking at the \nth{2} and \nth{5} parts, is the Langevin
equation that we wanted!

In the presence of an external force, this becomes

\begin{equation}
  m\fd{v}{t} = F_{ext} -\zeta v + \eta(t)
\end{equation}

and the slowly-varying part becomes

\begin{equation}
  m\fd{\avgb{v}}{t} = F_{ext} -\zeta \avgb{v}
\end{equation}
\section{Mean-square displacement}

%%The notation in clearer in this section if we use
%%$\avgb{x}$ to denote the average of $x$, rather than $\avgb{x}$.
As with the previous section, this follows \cite{ReifStatMech}
\S15.6. 
So, after having derived the equation of motion for our
particle 

\begin{equation*}
  m\fd{v}{t} = -\zeta v + \eta(t)
\end{equation*}
what sorts of quantities can we calculate? In absence of an external
force, it seems pretty clear that $\avgb{x} = 0$ in thermal
equilibrium, so
we'd better turn towards higher moments. The most commonly useful of
these is the mean-square displacement, $\avgb{x^2}$.

The key insight we'll need in order to find $\avgb{x^2}$ is that we
already know something about average values! In particular, the
equipartition theorem tells us that the kinetic energy,
$\frac{1}{2}mv^2$, ought to have average value $\frac{1}{2} kT$. If we
write $v$ as $\dot{x}$ and multiply both of those by two, and hope
that for our sanity mass does not fluctuate, we have

\begin{equation}
  \label{equipartition}
  m\avgb{\dot{x}^2} = kT
\end{equation}

If you haven't recently taken a probability/statistics class, two
reminders are in order. First, we need to be careful with averages, as
$\avgb{x}^2 \neq \avgb{x^2}$. Second, we often use the product rule to
do things like multiply both sides of

\begin{equation}
  m\fd{\dot{x}}{t} = -\zeta\dot{x} + \eta(t)
\end{equation}
by $x$ specifically so that we can get an equation involving
$\dot{x}^2$:

\begin{equation}
  \label{msd1}
  mx\fd{\dot{x}}{t}
  =  
  m\left[ \fd{}{t}(x\dot{x}) - \dot{x}^2 \right] 
  = 
  -\zeta x\dot{x} + x\eta(t)
\end{equation}

We're now in a position to take the ensemble average of both sides of
this equation. OK, a third reminder: the ensemble average of the sum
of two quantities is the sum of their respective ensemble averages. If
two quantities are independent, the ensemble average of their product
is the product of their ensemble averages. In particular, looking at
the right hand side of our equation,

\begin{equation}
  \avgb{-\zeta x\dot{x} + x\eta(t)} = \avgb{-\zeta x\dot{x}} + \avgb{x\eta(t)}
  = -\zeta \avgb{x\dot{x}} + \avgb{x}\avgb{\eta(t)}
  = -\zeta \avgb{x\dot{x}}
\end{equation}

where the last equality comes from the fact that the expectation value
of our randomly fluctuating force is, by definition, zero. If we
rearrange (\ref{msd1}) as

\begin{equation}
  \label{msd2}
  m\fd{}{t}(x\dot{x})
  = 
  m\dot{x}^2 -\zeta x\dot{x} + x\eta(t)
\end{equation}
we can take ensemble averages and use (\ref{equipartition}) to
write\footnote{OK, fine, a fourth thing to remember: you can commute
  the time derivative inside an ensemble average. To prove this to
  yourself, write the average out as an explicit sum and note that
  taking the derivative of the sum is just taking the sum of the
  derivatives of the individual terms.}

\begin{eqnarray}
  \avgb{m\fd{}{t}(x\dot{x})}
  &=&
  \avgb{m\dot{x}^2 -\zeta x\dot{x} + x\eta(t)} \\
  \nonumber
  m\fd{}{t}\avgb{x\dot{x}}
  &=&
  m\avgb{\dot{x}^2} -\zeta \avgb{x\dot{x}} \\
  \nonumber
  m\fd{}{t}\avgb{x\dot{x}}
  &=&
  kT -\zeta \avgb{x\dot{x}}\\
  \label{msd3}
  \fd{}{t}\avgb{x\dot{x}}
  &=&
  \frac{kT}{m} -\frac{\zeta}{m} \avgb{x\dot{x}}
\end{eqnarray}

That last line may look a little funny, but it's just an ODE in
$\avgb{x\dot{x}}$, which is something we surely know how to
solve:

\begin{equation}
  \label{msd4}
  \avgb{x\dot{x}} = Ce^{-\frac{\zeta}{m} t} + \frac{kT}{m}\frac{m}{\zeta}
  \equiv Ce^{-\frac{1}{\tau} t} + \frac{kT}{\zeta}
\end{equation}
where C is a constant of integration, and $\tau = \frac{m}{\zeta}$ is the
characteristic time. Before we plug in some initial conditions, note
that $\avgb{x\dot{x}}$ isn't a completely foreign quantity:

\begin{equation}
  \label{x2xxd}
  \frac{1}{2}\fd{\avgb{x^2}}{t} = \avgb{x\dot{x}}
\end{equation}
Now, as far as our initial conditions go, we want $\avgb{x^2}$ to be
the mean-square displacement of a particle, so it makes sense to set
$x(t=0) = 0$. Plugging that into (\ref{msd4}) gives


\begin{equation}
  0 = C + \frac{kT}{\zeta}
\end{equation}
so we know $C=-\frac{kT}{\zeta}$ and we can rewrite (\ref{msd4}) as 
\begin{equation}
  \label{msd5}
  \avgb{x\dot{x}} = \frac{1}{2}\fd{\avgb{x^2}}{t} = \frac{kT}{\zeta}(1-e^{-\frac{1}{\tau} t})
\end{equation}
and we can finally get an equation for $\avgb{x^2}$ by integrating
once:

\begin{equation}
  \label{msdfull}
  \avgb{x^2} = \frac{2kT}{\zeta}\left[t-\tau(1-e^{-\frac{1}{\tau} t})\right]
\end{equation}

%%\begin{wrapfigure}{l}{6cm}
\begin{tikzpicture}[scale=2]
    \draw[very thin,color=gray] (-0.1,-0.1) grid (2.1,2.1);
    \draw[->] (-0.2,0) -- (2.2,0) node[right] {$t$};
    \draw[->] (0,-0.2) -- (0,2.2) node[left] {$\avgb{x^2}$};
    \draw[domain=0:2] plot (\x,{\x - 0.25*(1-exp(-4*\x))}) node[below right] {};
\end{tikzpicture}
%%\end{wrapfigure}

At short time scales, that looks parabolic, and is called
``ballistic'' diffusion. At long time scales, that looks linear, and
is called ``normal'' diffusion. If we plotted this on a loglog scale,
we would indeed see two linear regions with slopes 2 and 1
respectively.

We can see this explicitly. If $t < \tau$, we can expand

\begin{equation}
  e^{-\frac{1}{\tau} t} = 1 - \frac{1}{\tau} t + \frac{1}{2}\frac{1}{\tau^2}t^2 + ...
\end{equation}
and we make our ``short'' time scale explicit as $t \ll \tau$, where
we can truncate the series as above and (\ref{msdfull}) becomes

\begin{equation}
  \label{msdshort}
  \avgb{x^2} \approx
  \frac{2kT}{\zeta}\left[t-\tau(1- 1 + \frac{1}{\tau} t - \frac{1}{2}\frac{1}{\tau^2}t^2 \right] 
  = kT\frac{1}{\tau\zeta} t^2 = \frac{kT}{m}t^2
\end{equation}
We can now see why this regime is called ``ballistic'': the particle
behaves as a free particle with constant velocity
$v=\sqrt{kT/m}$. Meanwhile, on long time scales $t \gg \tau$,
the exponential term in (\ref{msdfull}) tends towards zero, yielding 


\begin{equation}
  \label{msdlong}
  \avgb{x^2} = \frac{2kT}{\zeta}\left[t-\tau \right] 
  \approx \frac{2kT}{\zeta}t
\end{equation}
From your statistics classes, you should recognize this as the exact
same equation as a particle performing a random walk. Indeed, we
expect a random walker to diffuse with


\begin{equation}
  \avgb{x^2} = 2Dt
\end{equation}
in one dimension, so we've found an equation for our diffusion
constant,

\begin{equation}
  D = \frac{kT}{\zeta}
\end{equation}

The ``Stokes'' part of the picture involves classical
\textit{macroscopic} hydrodynamics showing that, for a sphere of
radius $a$ in a medium of viscosity $\eta$,

\begin{equation}
  \zeta  = 6\pi\eta a
\end{equation}
so we can write (\ref{msdlong}) as

\begin{equation}
  \avgb{x^2} = \frac{kT}{3\pi\eta a} t
\end{equation}
and we're finally ready to analyze some data! We take experimental
data, look at a plot of mean-square displacement vs. time, and extract
the diffusion constant from the slope. Knowing the viscosity of our
media and the temperature of the experiments then allows us to
determine Boltzmann's constant, all from watching grains of pollen
diffuse in water as Perrin did in his famous 1910 work.

\pagebreak
\section{Potential projects}
If this sort of thing strikes your fancy, you could certainly do an
independent project on related matters. Specific topics might include

\begin{itemize}
\item The generalized Langevin equation
\item Fluctuation-dissipation theorems
\item Fokker-Planck equations
\item Green-Kubo functions
\item Einstein's 1905 paper and Langevin's 1908 paper
\item Computer simulations of diffusion
\item Diffusion as a function of probability distributions
\end{itemize}



\section{Thanks}
 This lab is possible due to the
  generous support of Adam Hoppe at South Dakota State University.
\bibliographystyle{alpha}
\bibliography{diffusionbib}


\end{document}
