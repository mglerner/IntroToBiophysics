%% Document setup for Syllabus and HW

\textwidth=7in
\textheight=9.5in
\topmargin=-1in
\headheight=0in
\headsep=.5in
\hoffset  -.85in

\usepackage{amsmath} % for \text{}
\usepackage{amsfonts} % For \text{}
\usepackage{amssymb} % For \text{}
\usepackage{amsthm}
\usepackage{bigints}

%% Stolen from Nathan Baker
\newcommand{\erf}{{\mathrm{erf}}}
\newcommand{\erfc}{{\mathrm{erfc}}}
\newcommand{\erfi}{{\mathrm{erfi}}}
\newcommand{\argh}{{\mathrm{arg}}}
\newcommand{\atan}{{\mathrm{atan}}}
\newcommand{\acos}{{\mathrm{acos}}}
\newcommand{\asin}{{\mathrm{asin}}}
\newcommand{\mat}[1]{\,\underline{\underline{#1}}\,}
\newcommand{\abs}[1]{\left| #1 \right|}
\newcommand{\norm}[1]{\left\| #1 \right\|}
\newcommand{\order}[1]{{\mathcal{O}} \left( #1 \right)}
\newcommand{\op}[1]{{\mathcal{#1}}}
\newcommand{\myfile}[1]{\texttt{#1}}
\newcommand{\myvar}[1]{\textsf{#1}}
\newcommand{\mean}[1]{{\left\langle {#1} \right\rangle}}

% The master boolean -- set this to "masterfalse" or "mastertrue"
%\newif\ifmaster \masterfalse
\newif\ifmaster \mastertrue

\newcommand{\hidesolution}[1]{ \ifmaster { {#1} } \else { {Go to class.} } \fi }
\newcommand{\hideanswer}[1]{ \ifmaster { #1 } \else {} \fi }
\newcommand{\solution}[1]{ \begin{proof}[Solution] \hidesolution{#1} \end{proof} }
\newcommand{\answer}[1]{ \hideanswer{\begin{proof}[Answer] #1 \end{proof}} }

%% MGL
\newcommand{\CC}{\mathbb{C}}
\newcommand{\N}{\mathbb{N}}
\newcommand{\Ham}{\mathcal{H}}
\newcommand{\sL}{\mathcal{L}}
\newcommand{\sP}{\mathcal{P}}
\newcommand{\sV}{\mathcal{V}}

%\newcommand{\vect}[1]{\,\vec{\mathbf{#1}}\,}
\newcommand{\vect}[1]{\,\mathbf{#1}\,}
\usepackage{tabularx}
\newcommand{\vtau}{\vect{\tau}}
\newcommand{\vomega}{\vect{\omega}}
\newcommand{\va}{\vect{a}}
\newcommand{\vb}{\vect{b}}
\newcommand{\vc}{\vect{c}}
\newcommand{\vd}{\vect{d}}
\newcommand{\ve}{\vect{e}}
\newcommand{\vf}{\vect{f}}
\newcommand{\vg}{\vect{g}}
\newcommand{\vh}{\vect{h}}
\newcommand{\vi}{\vect{i}}
\newcommand{\vj}{\vect{j}}
\newcommand{\vk}{\vect{k}}
\newcommand{\vl}{\vect{l}}
\newcommand{\vm}{\vect{m}}
\newcommand{\vn}{\vect{n}}
\newcommand{\vo}{\vect{o}}
\newcommand{\vp}{\vect{p}}
\newcommand{\vq}{\vect{q}}
\newcommand{\vr}{\vect{r}}
\newcommand{\vs}{\vect{s}}
\newcommand{\vt}{\vect{t}}
\newcommand{\vu}{\vect{u}}
\newcommand{\vv}{\vect{v}}
\newcommand{\vw}{\vect{w}}
\newcommand{\vx}{\vect{x}}
\newcommand{\vy}{\vect{y}}
\newcommand{\vz}{\vect{z}}

\newcommand{\vA}{\vect{A}}
\newcommand{\vB}{\vect{B}}
\newcommand{\vC}{\vect{C}}
\newcommand{\vD}{\vect{D}}
\newcommand{\vE}{\vect{E}}
\newcommand{\vF}{\vect{F}}
\newcommand{\vG}{\vect{G}}
\newcommand{\vH}{\vect{H}}
\newcommand{\vI}{\vect{I}}
\newcommand{\vJ}{\vect{J}}
\newcommand{\vK}{\vect{K}}
\newcommand{\vL}{\vect{L}}
\newcommand{\vM}{\vect{M}}
\newcommand{\vN}{\vect{N}}
\newcommand{\vO}{\vect{O}}
\newcommand{\vP}{\vect{P}}
\newcommand{\vQ}{\vect{Q}}
\newcommand{\vR}{\vect{R}}
\newcommand{\vS}{\vect{S}}
\newcommand{\vT}{\vect{T}}
\newcommand{\vU}{\vect{U}}
\newcommand{\vV}{\vect{V}}
\newcommand{\vW}{\vect{W}}
\newcommand{\vX}{\vect{X}}
\newcommand{\vY}{\vect{Y}}
\newcommand{\vZ}{\vect{Z}}
%\newcommand{\pd}[2]{\frac{\partial#1}{\partial#2}}
%\newcommand{\dd}[2]{\frac{d#1}{d#2}}
\newcommand{\dinline}[2]{d #1/d #2} % for derivatives
\newcommand{\pdinline}[2]{\partial#1/\partial#2}

\newcommand{\boldcent}[1] {\begin{center}\textbf{ #1 }\end{center}}
% ***********************************************************
% ******************* PHYSICS HEADER ************************
% ***********************************************************
% From http://www.dfcd.net/articles/latex/latex.html
% Version 2
%\documentclass[11pt]{article} 
\usepackage{amsmath} % AMS Math Package
\usepackage{amsthm} % Theorem Formatting
\usepackage{amssymb}	% Math symbols such as \mathbb
\usepackage{graphicx} % Allows for eps images
\usepackage{multicol} % Allows for multiple columns
%\usepackage[dvips,letterpaper,margin=0.75in,bottom=0.5in]{geometry}
% % Sets margins and page size
%\pagestyle{empty} % Removes page numbers
%\makeatletter % Need for anything that contains an @ command 
%\renewcommand{\maketitle} % Redefine maketitle to conserve space
%{ \begingroup \vskip 10pt \begin{center} \large {\bf \@title}
%	\vskip 10pt \large \@author \hskip 20pt \@date \end{center}
%  \vskip 10pt \endgroup \setcounter{footnote}{0} }
%\makeatother % End of region containing @ commands
\renewcommand{\labelenumi}{(\alph{enumi})} % Use letters for enumerate
% \DeclareMathOperator{\Sample}{Sample}
\let\vaccent=\v % rename builtin command \v{} to \vaccent{}
\renewcommand{\v}[1]{\ensuremath{\mathbf{#1}}} % for vectors
\newcommand{\gv}[1]{\ensuremath{\mbox{\boldmath$ #1 $}}} 
% for vectors of Greek letters
\newcommand{\uv}[1]{\ensuremath{\mathbf{\hat{#1}}}} % for unit vector
\newcommand{\hx}{\uv{x}}
\newcommand{\hy}{\uv{y}}
\newcommand{\hz}{\uv{z}}
\newcommand{\hr}{\uv{r}}
\newcommand{\hphi}{\uv{\phi}}
\newcommand{\htheta}{\uv{\theta}}
%\newcommand{\abs}[1]{\left| #1 \right|} % for absolute value
% overbar taken from http://tex.stackexchange.com/questions/22100/the-bar-and-overline-commands
\newcommand{\overbar}[1]{\mkern 1.5mu\overline{\mkern-1.5mu#1\mkern-1.5mu}\mkern 1.5mu}
\newcommand{\avg}[1]{\overbar{#1}} % for average
\newcommand{\avgb}[1]{\left< #1\right>} % for average
%\newcommand{\avg}[1]{\left< #1 \right>} % for average

%\let\underdot=\d % rename builtin command \d{} to \underdot{}
%\renewcommand{\d}[2]{\frac{d #1}{d #2}} % for derivatives
\newcommand{\fd}[2]{\frac{d #1}{d #2}} % for derivatives
\newcommand{\dd}[2]{\frac{d^2 #1}{d #2^2}} % for double derivatives
\newcommand{\pd}[2]{\frac{\partial #1}{\partial #2}} 
% for partial derivatives
\newcommand{\pdd}[2]{\frac{\partial^2 #1}{\partial #2^2}} 
\newcommand{\pddx}[3]{\frac{\partial^2 #1}{\partial #2 \partial #3}} 
% for double partial derivatives
\newcommand{\pdc}[3]{\left( \frac{\partial #1}{\partial #2}
 \right)_{#3}} % for thermodynamic partial derivatives
\newcommand{\ket}[1]{\left| #1 \right>} % for Dirac bras
\newcommand{\bra}[1]{\left< #1 \right|} % for Dirac kets
\newcommand{\braket}[2]{\left< #1 \vphantom{#2} \right|
 \left. #2 \vphantom{#1} \right>} % for Dirac brackets
\newcommand{\matrixel}[3]{\left< #1 \vphantom{#2#3} \right|
 #2 \left| #3 \vphantom{#1#2} \right>} % for Dirac matrix elements
\newcommand{\grad}[1]{\gv{\nabla} #1} % for gradient
\let\divsymb=\div % rename builtin command \div to \divsymb
\renewcommand{\div}[1]{\gv{\nabla} \cdot #1} % for divergence
\newcommand{\curl}[1]{\gv{\nabla} \times #1} % for curl
\newcommand{\lap}[1]{\gv{\nabla}^2 #1} % for laplacian
\let\baraccent=\= % rename builtin command \= to \baraccent
\renewcommand{\=}[1]{\stackrel{#1}{=}} % for putting numbers above =
\newtheorem{prop}{Proposition}
\newtheorem{thm}{Theorem}[section]
\newtheorem{lem}[thm]{Lemma}
\theoremstyle{definition}
\newtheorem{dfn}{Definition}
\theoremstyle{remark}
\newtheorem*{rmk}{Remark}

% ***********************************************************
% ********************** END HEADER *************************
% ***********************************************************
